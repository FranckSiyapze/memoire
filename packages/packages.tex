\usepackage{graphics}    %permet d'inserer une image
\usepackage{float}  % permet de caler une image

\usepackage{shorttoc} %pour utiser le la commande\shorttableofcontents{Sommaire}{1} ////générer le sommaire

\usepackage{etoolbox}  %permet de numeroter les annexes en alpha{A,B.....ou a,b)

\usepackage[left=2.5cm,right=2.5cm,top=2.5cm,bottom=2.5cm]{geometry}

\usepackage[utf8]{inputenc}
\usepackage[T1]{fontenc}

\usepackage{amsmath,amsfonts,amssymb}

\usepackage{graphicx}

\usepackage[french]{babel}

\usepackage[table]{xcolor}
\frenchbsetup{StandardLists=true} % à inclure si on utilise \usepackage[french]{babel}

\usepackage{enumitem}

\usepackage{amssymb}

\usepackage{listings} % pour insérer du code source

\usepackage{times}

\usepackage{hyperref}
\hypersetup{pdfborder=0 0 0}

\usepackage{graphicx}

\graphicspath{ {images/} }

\usepackage{eso-pic} % nécessaire pour mettre des images en arrière plan

\usepackage{array}	

\usepackage{fancyhdr} % pour les en-têtes et pieds de pages

\usepackage{fancybox}

\usepackage{hyperref}

\usepackage{layout}

\usepackage{color}

\usepackage{tikz}

\usepackage{pifont}

\usepackage{titlesec}

\usepackage{color}

\usepackage{rotating}

%\usepackage[style=numeric]{biblatex}

\usepackage[Glenn]{fncychap} % pour customiser le style du titre \chapter

\usepackage{type1cm} % aussi pour les lettrines

\usepackage{lettrine}

\usepackage{longtable}
\usepackage{tabularx, blindtext}
\usetikzlibrary{calc}

\usetikzlibrary{decorations.pathmorphing}

\usepackage{array,multirow,makecell}

\usepackage{setspace}

\usepackage{wrapfig}

\usepackage{sidecap}

\setcellgapes{1pt}

\usepackage{array}

\usepackage{soul}
%\usepackage[nottoc]{tocbibind} % ajouter liste des figures et tableau dans sommaire

\pagestyle{fancy}


%%%%%%%%%%%%%%%%%%%style front%%%%%%%%%%%%%%%%%%%%%%%%%%%%%%%%%%%%%%%%% 


\renewcommand{\headrulewidth}{1pt}
%%% entête de page %%%%%
\fancyhead[C]{} 
\fancyhead[L]{\footnotesize{CONTRIBUTION A L'OPTIMISATION D'UNE APPLICATION DE MARKETPLACE CAS DE IZIWAY}}
\fancyhead[R]{}
%%% pied de page%%%%%
\renewcommand{\footrulewidth}{1pt}
\fancyfoot[C]{} 
\fancyfoot[L]{Rédigé et présenté par SIYAPZE Franck Dabryn}
\fancyfoot[R]{\thepage}



\renewcommand{\listfigurename}{Liste des figures}
 
\renewcommand{\listtablename}{Liste des tableaux}
                
                
                
                
                
%%%%%%%%%%%%%%%%%%% Définition des couleurs %%%%%%%%%%%%%%%%%%%
\definecolor{green1}{HTML}{556627}
\definecolor{green2}{HTML}{B7CA79}
\definecolor{green3}{HTML}{8FCF3C}
\definecolor{green4}{HTML}{1D702D}
\definecolor{green5}{HTML}{456B35}


\definecolor{blue1}{HTML}{1370e9}
\definecolor{blue2}{HTML}{2da9e9}

\chapter*{Conclusion}         % ne pas numéroter
\phantomsection\addcontentsline{toc}{chapter}{Conclusion} % dans TdM
\thispagestyle{fancy}
\vspace{-2cm}
Le travail effectué dans ce mémoire a eu pour objectif l’optimisation de l’application de la Marketplace IZIWAY CAMEROUN, dans le but d’améliorer sa qualité de service . C’est ainsi que nous avons analysé la problématique et nous sommes arrivé à optimiser l’application qui est une solution efficace et bénéfique. Pour cela on a mené en premier lieu une présentation des concepts généraux et du contexte. Ensuite, nous avons entamé le second chapitre dans lequel nous avons étudié l’existant et présenté un cahier des charges de la solution. Dans le troisième chapitre, nous avons décrit le langage de modélisation pour la conception de notre application qui est le langage UML. On a également recensé les acteurs qui interagissent avec l’application, puis, on a décrit les besoins de chaque acteur sous forme de cas d’utilisation. Et aussi, pour chaque cas d’utilisation, on a établi le diagramme de séquence dont l’objectif est de représenter les interactions entre les objets du système en indiquant la chronologie des échanges. Après, la réalisation d’une modèle statique représenté par le diagramme de classe. Enfin, on a pris le temps de réaliser à bien notre application tout en spécifiant les outils de développement ainsi que les langages de programmation utilisés, suivi d’un aperçu des interfaces que comprend celle-ci. Ce travail nous a permis d’acquérir une expérience personnelle et professionnelle. Il nous a été très bénéfique car on a eu la chance d’améliorer nos connaissances dans le domaine de la conception et cela sur le plan théorique, mais aussi de découvrir et d’acquérir de nouvelles connaissances en matière de programmation et de développement des bases de données en ce qui concerne l’aspect pratique. Nous souhaitons que ce travail puisse servir comme un outil d’aide et de documentation pour les étudiants à l’avenir, et une base de travail pour les utilisateurs concernés.
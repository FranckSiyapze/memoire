\chapter*{Résumé}                      % ne pas numéroter
\phantomsection\addcontentsline{toc}{chapter}{Résumé} % inclure dans TdM
\thispagestyle{fancy}
\vspace{-2cm}
%Nombre de mots: Souvent pas plus d’une page A4 selon les directives de votre %programme d'études

%Note:


%• N'écrivez le résumé qu'après la fin de la rédaction.


%• Utilisez le format ci-dessous pour un résumé bien structuré.


%• Assurez-vous que votre résumé montre clairement en quoi consiste le document %en général (également pour les personnes sans connaissances préalables).

%Et ...!


%• N'utilisez pas d'exemples.


%• Ne présentez pas de nouvelles informations.


%• Ecrire court et percutant.

%début para 1%


%Paragraphe 1
%Description du problème
%- Quel est le problème?
%- Quel est l’objectif?
%- Quelle est la question principale? Si vous avez des hypothèses, présentez-les ici.
%N'écrivez pas toutes vos sous-questions ici.
Le terme UX (acronyme de l’anglais : User Expérience), expérience utilisateur en français, désigne la qualité de l’expérience vécue par l’utilisateur dans toute situation d’interaction. L’objectif primordial de notre stage a été de développer une solution informatique permettant d’augmenter la qualité de service de l’application IZIWAY CAMEROUN. Cette solution vient pallier à des problèmes tels que le faible taux d’utilisateur, le faible taux de conversion ( nombre d’achat) sur la plateforme, le manque d’animation sur l’application. Au regard de ces différents problèmes, nous avons travaillé sur comment optimiser l’application mobile d’une entreprise qui fait dans le E-commerce. Dans cette optique, nous avons opté pour la méthode agile plus précisément la méthodologie SCRUM en raison de la complexité du projet et surtout parce que cette méthodologie de travail met le client au centre des activités.  L’application IZIWAY CAMEROUN est basée sur l’architecture MVC, l’ensemble des interfaces ont été développés avec le Framework Flutter, les interactions avec la base de données stockées dans le système de gestion de base de données MYSQL SERVER se font à travers une api qui a été développé en Csharp. Au vu des éléments qui précèdent, nous pouvons dire qu’il y’a une nette amélioration sur l’ergonomie de l’application ce qui participe à l’augmentation du nombre d’utilisateur et augmente le taux de conversion (taux d’achat) sur notre plateforme. La solution offerte à la suite de notre travail est une réussite en raison de la satisfaction du top management de IZIWAY CAMEROUN et de celle des employés qui n’ont pas manqué de signifier une nette amélioration dans leur travail.\\


\textit{Mots clés} : Marketplace, Expérience utilisateur, Processus d’achat.
\chapter*{Introduction}                      % ne pas numéroter
\phantomsection\addcontentsline{toc}{chapter}{Introduction} % inclure dans TdM
\thispagestyle{fancy}
\vspace{-2cm}

Le concept de e-commerce a fait son apparition au moment des premières transactions faites par des moyens électroniques. Dans les années 90, internet fait son apparition dans les familles Française lorsque les ordinateurs se commercialisent et se démocratisent. Les transactions entres entreprises et particuliers  commencent alors apparaître. En 1997, le e-commerce se démocratise enfin et les grandes entreprises du secteur informatique comme Microsoft commencent à entrer dans le virtuelle. Les nouvelles technologies (smartphones, tablettes,...) dont nous disposons maintenant, nous permettent de passer de la vie réelle a la vie virtuelle en quelques instants, ainsi les consommateurs ont vu dans ce nouveau canal un moyen de trouver le meilleur rapport qualité prix. Pour pouvoir permettre une gestion optimale d’une entreprise de e-commerce, les différents services s’appuient sur les processus métier tel que les passations de commandes, les demandes de publications des articles. Ces processus exécutés de manière manuelle ou même sémi-manuelle ( via les outils développés), rencontrent plusieurs problèmes à savoir : le faible taux d’utilisateur sur l’application, le faible taux de conversion ( nombre d’achat ) des articles à travers l’application, le manque d’animation sur l’application. A travers ses problèmes cités, nous ressortons la problématique suivante : « Comment améliorer l’expérience utilisateur et augmenter notre chiffre d’affaire ». L’objectif majeur de notre étude est d’assurer la qualité de services de l’application. Les objectifs spécifiques sont : améliorer l’ergonomie de l’application, augmenter le nombre d’animation sur l’application, améliorer les indicateurs de performance ( taux de conversion, ).La première chapitre et deuxième chapitre porterons sur l'etude de l'art c’est-à-dire les notions d'une optimisation, ensuite nous ferons une étude de l’existant c’est-à-dire présenter la Marketplace les différentes typologie de Marketplace qui existe. Ensuite nos deux derniers, nous ferons une étude, une analyse, une conception et implémentation de la solution répondant à la problématique étudiée.\\
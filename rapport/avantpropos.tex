\chapter*{Avant-propos}         % ne pas numéroter
\phantomsection\addcontentsline{toc}{chapter}{Avant-propos} % inclure dans TdM
\thispagestyle{fancy}
Dans le but d’assurer un développement durable et de fournir aux entreprises une main d’œuvre compétente et compétitive dans divers domaines, le gouvernement Camerounais par le biais de Ministère de l’enseignement Supérieur a permis l’ouverture des Instituts Privés d’enseignement Supérieur. Donnant ainsi l’opportunité aux institutions privées, de contribuer à l’acquisition d’une formation académique et professionnelle en adéquation avec le monde professionnel.

C’est ainsi qu’est crée l’ISTDI (Institut Supérieur de Technologie et du Design Industriel) par arrêté Ndeg \textbf{02/0094/MINESUP/DDES/ESUP}  du \textbf{13 septembre 2002}  et autorisation d’ouverture Ndeg \textbf{0102/MINESUP/DDES/ESUP}  du \textbf{18 septembre 2002.}  Située dans la région du littoral, département du Wouri, l’arrondissement de Douala 5ème, au quartier Logbessou. L’ISTDI est ensuite érigée en \textbf{Institut Universitaire de la Côte (IUC)} par arrêté Ndeg \textbf{5/05156/N/MINESUP/DDES/ESUP/SAC/ebm,} et comporte à ce jour quatre (04) établissements notamment :

\begin{enumerate}
	\item \textbf{L’Institut Supérieur de Technologies et du Design Industriel (ISTDI)} qui	forme dans les cycles et filières suivants :
	
	\begin{itemize}
		\item \textbf{BTS Industriels :}
		
		\begin{itemize}
			\item Maintenance des systèmes informatiques (MSI)
			\item Informatique Industrielle (II)
			\item Electrotechnique (ET)
			\item Electronique (EN)
			\item Froid et climatisation (FC)
			\item Maintenance et après vente automobile (MAVA)
			\item Génie Civil (GC)
			\item Génie Bois (GB)
			\item Chaudronnerie (CH)
			\item Fabrication Mécanique (FM)
		\end{itemize}
		
		\item \textbf{LICENCES PROFESSIONNELLES INDUSTRIELLES et TECHNOLOGIQUES en partenariat avec l’université de Dschang :}
		
		\begin{itemize}
			\item Administration et Sécurité des Réseaux
			\item Génie Logiciel
			\item Automatique et Informatique Industrielle
			\item Electrotechnique
			\item Electronique
			\item Management des Services Automobiles
			\item Maintenance et Expertise des Automobiles
			\item Maintenance des Systèmes Industriels
			\item Génie Energétique et Industriel
			\item Génie Civil
			\item Génie Bois
			\item Génie Mécanique et Productique
		\end{itemize}
		
		\item \textbf{MASTER PROFESSIONNEL INDUSTRIEL en partenariat avec l’université de Dschang :}
		
		\begin{itemize}
			\item Génie Electrique et Informatique Industrielle
			\item Génie Télécommunications et Réseaux
			\item Systèmes d’Information et Génie Logiciel
		\end{itemize}
	\end{itemize}
	
	\item \textbf{L’Institut de Commerce et d’Ingénierie d’Affaires (ICIA)} qui forme dans les cycles et filières suivantes :
	
	\begin{itemize}
		\item \textbf{BTS Commerciaux :}
		\begin{itemize}
			\item Assurance(AS)
			\item Informatique de Gestion (IG)
			\item Banque et Finance(BF)
			\item Action Commerciale (ACO)
			\item Commerce International (CI)
			\item Communication d’Entreprise (CE)
			\item Comptabilité et Gestion des Entreprises (CGE)
			\item Logistique et Transport (LT)
		\end{itemize}
		\item \textbf{LICENCES PROFESSIONNELLES COMMERCIALES et GESTION
			en partenariat avec l’université de Dschang :}
		\begin{itemize}
			\item Marketing
			\item Finance-Comptabilité
			\item Banque
			\item Banque - Assurance
			\item Gestion des Ressources Humaines
			\item Logistique et Transport
			\item Contrôle et Audit
			\item Gestion Qualité
		\end{itemize}
		\item \textbf{MASTER PROFESSIONNEL COMMERCIAL en partenariat avec l’université de Dschang :}
		\begin{itemize}
			\item Management des organisations
			\item Finance-Comptabilité
		\end{itemize}
		\item \textbf{MASTER ISUGA-France en partenariat avec EMBA France}
	\end{itemize}
	
	
	\item \textbf{L’Institut d’Ingénierie Informatique d’Afrique Centrale (3IAC)} qui forme dans les cycles et filières suivantes :
	
	\begin{itemize}
		\item \textbf{Cycle des TIC en partenariat avec CCNB-DIEPPE du Canada :}
		\begin{itemize}
			\item Réseautique et sécurité
			\item Programmation et Analyse
		\end{itemize}
		\item \textbf{MASTER EUROPEEN en partenariat avec 3IL en France :}
		\begin{itemize}
			\item Génie Logiciel
			\item Administration des Systèmes Réseaux
		\end{itemize}
		\item \textbf{MASTER PROFESSIONNEL en partenariat avec l’ENSP Yaoundé :}
		\begin{itemize}
			\item Génie Civil
			\item Génie Energétique et Environnement
			\item Génie Industriel et Maintenance
		\end{itemize}
		\item \textbf{Cycle des Classes Préparatoires aux Grandes Ecoles d’Ingénieurs (CP) en partenariat avec l’université du Maine en France :}
		\begin{itemize}
			\item Classes préparatoires
			\item Licences Sciences et Techniques
		\end{itemize}
		\item \textbf{CYCLE INGENIEUR de Génie Industriel (ESSTIN-France)}
		\item \textbf{Pôle de Recherche Innovation et Entreprenariat (PRIE)}
	\end{itemize}

\end{enumerate}

L’étudiant du cycle Ingénieur/Master est tenu en Cinquième année d’effectuer un stage de six mois en entreprise pour la fin de son cursus. Ce stage permettra aux étudiants de mieux appréhender le monde professionnel et de compléter les connaissances acquises. C’est dans cette optique que nous avons effectué un stage au sein de l’entreprise PRIMA ASSURANCES ; où nous avons travaillé sur le thème \textbf{«GESTION DES SINISTRES»}
\documentclass[a4paper,11pt,final]{article}
% Pour une impression recto verso, utilisez plutôt ce documentclass :
%\documentclass[a4paper,11pt,twoside,final]{article}

\usepackage[english,francais]{babel}
\usepackage[utf8]{inputenc}
\usepackage[T1]{fontenc}
\usepackage[pdftex]{graphicx}
\usepackage{setspace}
\usepackage{hyperref}
\usepackage[french]{varioref}
\newcommand{\reportsubject}{JouRs 2014 (Journ\'ees de Recherches du PRIE)}
\newcommand{\reporttitle}{R\'esum\'e}     % Titre
\newcommand{\reportauthor}{Bruno \textsc{Voisin}} % Auteur
%\newcommand{\reportsubject}{Stage de fin d'étude} % Sujet
\newcommand{\HRule}{\rule{\linewidth}{0.5mm}}
\setlength{\parskip}{1ex} % Espace entre les paragraphes

\hypersetup{
    pdftitle={\reporttitle},%
    pdfauthor={\reportauthor},%
    pdfsubject={\reportsubject},%
    pdfkeywords={rapport} {vos} {mots} {clés}
}

\begin{document}
  \include{title}
  \cleardoublepage % Dans le cas du recto verso, ajoute une page blanche si besoin
  \tableofcontents % Table des matières
  \sloppy          % Justification moins stricte : des mots ne dépasseront pas des paragraphes
  \cleardoublepage
  \chapter*{REMERCIEMENTS}         % ne pas numéroter
\phantomsection\addcontentsline{toc}{chapter}{Remerciements} % inclure dans TdM
\thispagestyle{fancy}
\vspace{-2cm}

Avant de présenter ce travail effectué dans le cadre du projet de fin d’études Ingénieur Informatique modules : Java Professionnel et Technologie web et mobile, nous tenons à remercier tous ceux et celles qui ont participé à la réussite de notre projet, et ceux qui ont fait de notre formation une expérience enrichissante. Nos remerciements vont à l’endroit de :

\begin{itemize}
	\item Monsieur GUIMEZAP Paul, Président Fondateur de l’IUC pour l’initiative qu’il a eu de contribuer à la formation de la jeunesse;
	
	\item Madame NOUBANKA Manuella, Directrice de 3IAC pour son encadrement, ses conseils et l'attention qu’elle nous a accordé;
	
	\item Docteur AZEUFACK ULRICH, le Superviseur, pour ses conseils et son suivi tant sur le plan académique que professionnel ; 
	
	\item Monsieur TEKOUDJOU Xavier, l’encadreur Académique, qui de ses mains de maitre, a su canaliser ce présent travail ; 
	
	\item Monsieur TALOM GUY, Le CEO de IZIWAY CAMEROUN ,de nous avoir accordé cette confiance d’effectuer notre stage ; 
	
	\item Madame SIMO CLARISSE EPSE SIYAPZE, La Directrice GENERALE de IZIWAY CAMEROUN, pour ses encouragements et ses conseils stratégiques sur le projet ; 

	\item Monsieur TCHOUA EDMOND  l’encadreur professionnel pour ses multiples conseils et l’édification de la vie en entreprise ;
	
	\item L’équipe pédagogique de notre école 3IL, pour nous avoir apporté les connaissances nécessaires nous permettant d’effectuer notre travail dans les meilleures conditions.
	
\end{itemize}

  \cleardoublepage
  \include{intro}
  \cleardoublepage
  \include{partie1}
  \cleardoublepage
  \include{partie2}
  \cleardoublepage
  \include{concl}
  \cleardoublepage
  \include{references}
\end{document}
